\section*{Règlement 2 - Relatif à l'AG} 
\addcontentsline{toc}{section}{Règlement 2 - Relatif à l'AG}

\subsection*{Article I. Définition de l'AG}
\addcontentsline{toc}{subsection}{Article I. Définition de l'AG}
Réunion des membres réguliers de la promotion afin de prendre certaines décisions pour le fonctionnement de celle-ci. 

\subsection*{Article II. La convocation}
\addcontentsline{toc}{subsection}{Article II. La convocation}
Les assemblées générales des membres sont convoquées par la présidence du CE-A. De plus, si vingt (20) personnes ou plus signent une demande officielle de convocation, la promotion devra tenir une AG dans les deux (2) semaines qui suivront la remise recommandée de l’avis au bureau de l’AGEG (C1-2045).

\subsection*{Article III. Convocation des assemblées générales}
\addcontentsline{toc}{subsection}{Article III. Convocation des assemblées générales}

\subsubsection*{Section 3.01 : Convocation d'une AG normale}
\addcontentsline{toc}{subsubsection}{Section 3.01 : Convocation d'une AG normale}
La convocation d’une AG normale se déroule comme suit :
\begin{enumerate}
\item La décision : Le CE actif se réunit pour évaluer la date, l’heure et le lieu de l’AG. Un vote au 2/3 clair des membres du CE est nécessaire pour valider ces informations.
\item L’annonce : Le CE-A annonce par, sans s’y limiter, l’AGEGNews et par courriel aux étudiants (liste d’envoi du registre des membres), au minimum une semaine avant la date de l’assemblée, la date, l’heure et le lieu officiel sélectionnés pour l’AG; ces informations deviennent alors non modifiables. Les membres ont alors quarante-huit (48) heures pour indiquer à un membre du CE-A des empêchements majeurs et demander un changement de date. Dans l’éventualité où un événement mettrait en péril l’atteinte du quorum, le CEA se doit de réévaluer le moment de tenue de l’assemblée.
\item L’envoi : Le CE envoie l’ordre du jour de l’AG au minimum quarante-huit (48) heures à l’avance par courriel aux étudiants (liste d’envoi du registre des membres).
\end{enumerate}

\subsubsection*{Section 3.02 : Convocation d'une AG spéciale}
\addcontentsline{toc}{subsubsection}{Section 3.02 : Convocation d'une AG spéciale}
La présidence de la promotion peut, si les circonstances le justifient, appeler une AG spéciale. Cette assemblée doit être convoquée au minimum quarante-huit (48) heures à l’avance. L’assemblée devra premièrement valider sa tenue en acceptant les raisons de la présidence à majorité absolue, soit cinquante pour cent plus une voix (50 \% + 1 voix), au début de l’assemblée. De plus, l’ordre du jour doit être envoyé par courriel (liste d’envoi du registre des membres) au moins vingt-quatre (24) heures à l’avance à tous les membres.
\\

Un membre peut, s’il le désire, demander la convocation d’une AG spéciale. Celui-ci doit recueillir un minimum de 30 signatures de membres de la promotion 61, dont au moins deux personnes par programme actif. Une fois le nombre de signatures nécessaires recueilli, le membre doit transmettre la liste de signatures, l’ordre du jour, le local, la date et l’heure de la réunion au CEA. Celui-ci est responsable de divulguer les informations selon les modalités définies ci-haut.

\subsection*{Article IV. Le quorum}
\addcontentsline{toc}{subsection}{Article IV. Le quorum}
Le quorum est fixé selon la règle suivante : vingt (20) membres actifs de la 61$^e$ promotion ainsi que la majorité des membres du CE-A, dont au moins  une (1) personnes de chaque programme actif.

\subsection*{Article V. Présidence d'assemblée}
\addcontentsline{toc}{subsection}{Article V. Présidence d'assemblée}
La présidence de l’assemblée sera tenue par une personne extérieure à la promotion, choisie pour ses compétences dans la tenue d’AG. Cette personne doit être approuvée par l’AG. En cas de refus de la présidence d’assemblée proposée lors de l’AG, l’assemblée devra proposer un remplaçant satisfaisant les critères cités précédemment.

\subsection*{Article VI. Secrétariat d'assemblée}
\addcontentsline{toc}{subsection}{Article VI. Secrétariat d'assemblée}
Le secrétariat de l’assemblée sera tenu par une personne extérieure à la promotion choisie pour ses compétences dans la rédaction de procès-verbaux. Cette personne devra être approuvée par l’AG. En cas de refus du secrétariat d’assemblée proposé lors de l’AG, l’assemblée devra proposer un remplaçant satisfaisant les critères cités précédemment.

\subsection*{Article VII. Élections}
\addcontentsline{toc}{subsection}{Article VII. Élections}
Les élections de candidats pour les postes de la finissante à combler (CE + Comité 5@8) seront faites de manière électronique, avec une plateforme web permettant le vote unique. Seuls les membres de la promotion peuvent voter. Le formulaire de vote est transmis par courriel aux membres inscrits sur le registre. Le vote sera ouvert à partir de 2h00 après la fermeture de l’AG, où les membres auront la chance de se présenter de vive voix (1 minute de présentation et 1 minute de questions par membre par poste), et pour une période d'au moins 6h. La procédure de vote doit permettre aux candidats membres de la promotion de s’inscrire à au plus 3 postes, avec un choix de préférence (exemple : 1 à 3 pour 3 postes A,B,C : A=1 , B=3 , C=2 ) pour chaque poste. Aussi, lors du vote, les membres doivent pouvoir voter en indiquant leur choix de préférence pour les candidats à chaque poste (exemple : pour le poste A, candidats X, Y, Z : X=1, Y=3 , Z=2) avec un maximum de vote pour 3 candidats possible. 
Lors de la compilation des votes, un système de pointage sera utilisé :  
\begin{itemize}
\item Les premières, secondes et troisième place valent respectivement quatre, deux et un point;
\item Un candidat est élu si ce dernier a plus de points que les autres candidats et la chaise;
\item Si plusieurs candidats ont le même pointage pour un poste, celui qui a le plus de votes de premier choix sera classé avant les autres;
\item Si plusieurs candidats ont le même pointage et le même nombre de votes de premier choix, celui qui a le plus de votes de deuxième choix sera classé avant les autres;
\item Si plusieurs candidats ont le même pointage, le même nombre de votes de premier choix et le même nombre de votes de deuxième choix, un vote en AG sera refait;
\item Si la chaise a plus de points que les autres candidats, le poste reste inoccupé;
\item Si un candidat est élu à plus qu'un poste, le poste pour lequel il a indiqué la préférence la plus forte lui sera attribué en priorité;
\end{itemize}

\subsection*{Article VIII. Le vote}
\addcontentsline{toc}{subsection}{Article VIII. Le vote}
Tous les membres réguliers ont droit à un vote par membre. Les votes par procuration doivent être approuvés par l’assemblée. Seule une raison de force majeure peut justifier un vote par procuration et une lettre signée par le membre est requise. À toutes les assemblées générales, les votes se font selon la méthode décrite dans l'Article VII.\\
Le vote se fait par majorité simple, en cas d'égalité, un second tour de vote par scrutin secret a lieu uniquement avec les membres présents en AG. En cas d’égalité lors d’une élection, la marche à suivre sera décidée par l’AG.

\subsection*{Article IX. La conduite des assemblées générales}
\addcontentsline{toc}{subsection}{Article IX. La conduite des assemblées générales}
Toutes les assemblées générales sont soumises aux règles du code Morin sauf sous indication contraire de ce règlement (voir Annexe). 

\subsection*{Article X. Les rôles et pouvoirs}
\addcontentsline{toc}{subsection}{Article X. Les rôles et pouvoirs}

\subsubsection*{Section 10.01 : Les rôles et pouvoirs ordinaires}
\addcontentsline{toc}{subsubsection}{Section 10.01 : Les rôles et pouvoirs ordinaires}
L’assemblée a comme pouvoirs ordinaires :
\begin{itemize}
\item L’élection des membres du CE;
\item L’adoption du bilan financier de la session en cours;
\item L’adoption du bilan financier annuel.
\end{itemize}
 
Les décisions qui découlent de ces pouvoirs sont votées à la majorité simple par les membres actifs présents à l’AG.

\subsubsection*{Section 10.02 : Les rôles et pouvoirs spéciaux}
\addcontentsline{toc}{subsubsection}{Section 10.02 : Les rôles et pouvoirs spéciaux}
L’assemblée a comme pouvoirs spéciaux :
\begin{itemize}
\item La modification de la charte;
\item La modification des règlements;
\item La modification du logo;
\item La prise de prêt par la promotion;
\item La suspension et l'exclusion de membres;
\item La dérogation d’un règlement de promotion;
\item Le renversement d'une décision prise par le CE-A;
\item La destitution d’un membre du CE-A avec un argumentaire valable;
\item Le recours relevant du domaine légal municipal, provincial et/ou fédéral.
\end{itemize}

Les décisions qui découlent de ces pouvoirs sont votées à majorité renforcée aux deux tiers (2/3) par les membres actifs présents à l’AG.\\

Les décisions qui découlent de ces pouvoirs sont votées au double vote, qui est défini à l’article XI du présent règlement.

\subsection*{Article XI. Le double vote}
\addcontentsline{toc}{subsection}{Article XI. Le double vote}
Le double vote est nécessaire pour les décisions découlant des pouvoirs spéciaux prises en AG.\\

Le double vote permet aux étudiants passifs de prendre position lors d’un vote.\\

Les propositions nécessitant un double vote seront votées selon la procédure qui suit :
\begin{enumerate}
\item Si cette proposition est adoptée en AG, elle est présentée par courriel à tous les étudiants de la promotion par l’entremise du procès-verbal et si aucune opposition ne se présente de la part de 12 étudiants passifs dans les 7 jours suivants l’envoi du courriel, la proposition sera adoptée.
\item Dans le cas où une opposition est soulevée par un minimum de 12 étudiants passifs, la proposition sera ramenée au vote lors de la première AG de la session qui suit et devra être approuvée; le CE-A doit s’assurer que les personnes qui votent ne l’ont pas fait à la session précédente.
\item Dans le cas d’une dissension entre les deux votes, un vote par courriel sera effectué par la présidence de promotion.
\end{enumerate}

\subsection*{Article XII. Procès-verbal}
\addcontentsline{toc}{subsection}{Article XII. Procès-verbal}
Le procès-verbal de l’AG est un document public. Ce document sera disponible sur le site web de la promotion le plus tôt possible après la tenue de l’assemblée ainsi que dans le dossier correspondant sur le nuage de l’AGEG.

\subsection*{Article XIII. Conflits d'intérêts}
\addcontentsline{toc}{subsection}{Article XIII. Conflits d'intérêts}
Un conflit d'intérêts est une situation dans laquelle une personne ayant à accomplir une fonction d'intérêt général a des intérêts personnels en concurrence avec la mission qui lui est confiée, l’intérêt de la promotion ou celle de l’AGEG. De tels intérêts en concurrence peuvent mettre cette personne en difficulté pour l’accomplissement de sa tâche de façon neutre et impartiale.\\

Même s'il n'y a aucune preuve d'actes préjudiciables, une apparence de conflit d'intérêts peut engendrer une situation pouvant mettre en doute la confiance en cette personne en ce qui a trait à l’acceptation de ses responsabilités.\\

Une personne dans cette situation devra annoncer son conflit d’intérêts et pourra exercer son droit de parole avant de quitter la salle dans l'éventualité où un vote sur le sujet est requis.

\subsection*{Article XIV. Création et abolition de postes}
\addcontentsline{toc}{subsection}{Article XIV. Création et abolition de postes}
L’AG peut, lorsqu’elle le juge utile, créer d’autres postes et nommer, pour les occuper, les mandataires qu'elle juge à propos. Les mandataires nommés exerceront les pouvoirs et rempliront les fonctions et devoirs que l'AG pourra leur transférer par résolution. L'AG a aussi le pouvoir d'abolir ou de fusionner des postes lorsque ce sera opportun. Il sera également du ressort de l’AG d’attribuer la rémunération du poste au besoin.

