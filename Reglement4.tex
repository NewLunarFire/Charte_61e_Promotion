\section*{Règlement 4 - Relatif au CE} 
\addcontentsline{toc}{section}{Règlement 4 - Relatif au CE}

\subsection*{Article I. La composition du CE non finissant}
\addcontentsline{toc}{subsection}{Article I. La composition du CE non finissant}
Pour faire partie du CE, il faut être membre actif de la promotion. Le CE est composé des membres suivants : 
\begin{itemize}
\item Présidence; 
\item Vice-présidence et direction Affaires Sociales;
\item Trésorerie; 
\item secrétariat; 
\item Représentants de programmes actifs.
\end{itemize}

\subsubsection*{Section 1.01 : Représentants de programmes actifs}
\addcontentsline{toc}{subsubsection}{Section 1.01 : Représentants de programmes actifs}
Les postes de représentant de programme actif sont pourvus par des étudiants membres de la promotion et qui étudient dans le programme d’études pour lequel ils veulent être représentants. Seuls les programmes actifs ont droit à un représentant de ce type. Ces représentants sont des membres normaux du conseil et ont tous les droits, pouvoirs et responsabilités normales. Un représentant est élu en AG par un vote à majorité. Seuls les membres de la promotion qui étudient dans sa programme  et qui sont dans l’alternance active peuvent voter. Un membre peut être candidat à ce poste autant de fois qu’il le désire. 

\subsection*{Article II. La composition du CE finissant}
\addcontentsline{toc}{subsection}{Article II. La composition du CE finissant}
Le CE finissant entre en fonction au début de la session où ils deviennent finissants.\\

Pour faire partie du CE finissant, il faut être membre régulier actif de la promotion (sauf Direction Journée Carrière). Le CE finissant est composé des membres suivants :
\begin{itemize}
\item Présidence; 
\item Vice-présidence; 
\item Trésorerie; 
\item Secrétariat;
\item Webmestre; 
\item 2 directions 5@8;
\item Direction Affaires sociales; 
\item Direction Bal / Voyage; 
\item Direction Souvenirs;
\item Direction Casino / Jonc; 
\item Direction Journée Carrière;
\item Représentants de programme  (si nécessaire).
\end{itemize}

\subsubsection*{Section 2.01 : Représentants de programmes actifs}
\addcontentsline{toc}{subsubsection}{Section 2.01 : Représentants de programmes actifs}
Le poste de représentant de programme actif a pour but de s’assurer que toutes les programmes sont présentes au sein du CE, omit la présidence, la vice-présidence, la trésorerie, le secrétariat et le webmestre. Une fois que l’élection de toutes les directions a été effectuée et que l’AG réalise l’absence d’un programme actif chez les directions, l’AG doit alors nommer autant de personnes que nécessaires pour que toutes les programmes soient représentées. Ces élus seront alors des représentants de programme  et auront tous les droits, pouvoirs et responsabilités d’un membre du CE. Seuls les membres dudit programme peuvent voter lors de l’élection d’un représentant de programme.

\subsection*{Article III. CE actif et passif}
\addcontentsline{toc}{subsection}{Article III. CE actif et passif}

\subsubsection*{Section 3.01 : CE actif (CE-A)}
\addcontentsline{toc}{subsubsection}{Section 3.01 : CE actif (CE-A)}
Le CE-A est celui qui prend les décisions courantes de la promotion. Le mandat du CE-A va du début à la fin des activités pédagogiques de la session pour laquelle il a été élu. Il devient alors le CE-P.

\subsubsection*{Section 3.02 : CE passif (CE-P)}
\addcontentsline{toc}{subsubsection}{Section 3.02 : CE passif (CE-P)}
Le CE-P est le conseil qui s’occupe de transmettre les informations aux étudiants non présents. Les membres du CE-P ont accès en tout temps à toutes les réunions du CE-A, même les réunions sous huis clos. Son mandat va de la fin des activités pédagogiques de la session où il a été élu à la fin des activités pédagogiques de la session suivante. Le dernier CE-P est le CE-A  de l’été 2019.

\subsection*{Article IV. L'élection du CE}
\addcontentsline{toc}{subsection}{Article IV. L'élection du CE}

\subsubsection*{Section 4.01 : Élection du CE-P}
\addcontentsline{toc}{subsubsection}{Section 4.01 : Élection du CE-P}
Le CE-P est le CE-A de la session précédente, et ce, dans le but d’assurer une continuité dans les dossiers de la promotion.

\subsubsection*{Section 4.02 : Élection du CE-A}
\addcontentsline{toc}{subsubsection}{Section 4.02 : Élection du CE-A}
L’élection des membres du CE-A se fait lors des sessions mentionnées dans le tableau ci-dessous. Lorsque l’élection est au début de la session, il est du mandat du CE-P d’organiser la tenue de l'élection. Celle-ci doit avoir lieu dans les deux premières semaines de la session. Il est de la responsabilité du CE-A d’organiser l’élection lorsqu’elle est faite dans une session précédente.

\begin{table}[H]
    \centering
    \caption{Moments des élections pour chaque session}
    \begin{tabular}{|l|l|}
      \hline
      Session & Moment de l'élection \\
      \hline
      Automne 2015 et Hiver 2016 & Automne 2015 \\
      \hline
      Été 2016 & Été 2016 \\
      \hline
      Automne 2016 & Automne 2016 \\
      \hline
      Hiver 2017 & Été 2016 \\
      \hline
      Été 2017 & Été 2017 \\
      \hline
      Automne 2017 & Hiver 2017 \\
      \hline
      Hiver 2018 & Été 2017 \\
      \hline
      Été 2018 & Été 2018 \\
      \hline
      Automne 2018 & Hiver 2018 \\
      \hline
      Hiver 2019 & Été 2018 \\
      \hline
      Été 2019 & Automne 2018 \\
      \hline
      Automne 2019 & Automne 2019 \\
      \hline
    \end{tabular}
\end{table}

\textit{*Le CE-A de la session du moment de l'élection est responsable d'organiser une AG dès le début de la session en cours pour élire le nouveau CE-A si les échéanciers ci-dessus n'ont pas été respectés.}

\subsection*{Article V. Poste vacant sur le CE-A}
\addcontentsline{toc}{subsection}{Article V. Poste vacant sur le CE-A}
Pour tout poste laissé vacant sur le CE-A, le CE-A peut nommer par intérim une personne afin de pourvoir au poste. Le(s) poste(s) vacant(s) doit(doivent) être annoncé(s) par courriel à l'ensemble de la promotion et sur le groupe Facebook de la promotion, sans s’y limiter.La moitié des postes du CE-A doivent être comblés en AG, sans quoi une seconde AG est nécessaire pour le constituer.

\subsection*{Article VI. Les rôles et pouvoirs du CE-A}
\addcontentsline{toc}{subsection}{Article VI. Les rôles et pouvoirs du CE-A}
\begin{itemize}
\item Veiller et décider aux affaires courantes de la promotion dont il est le porte-parole et le représentant.
\item Informer la présidence du CE-P par courriel de toute dépense supérieure à cinq cents (500) dollars. Dans ce courriel, la trésorerie doit présenter le bilan financier mis à jour, sauf si cette dépense est reliée au 5@8, aux activités 5@8 prolongées ou à une activité de financement. Le CE-P dispose de 24h pour s’opposer à une demande de dépense de la part du CE-A.
\item Tous les investissements autorisés par le CE doivent être justifiés devant l'AG et entérinés par celle-ci. 
\item Veiller à l'exécution des décisions de l'AG.
\item Si nécessaire, le CE-A peut s'adjoindre des personnes et former des groupes de travail pour la conduite des affaires de la promotion. 
\item Le CE-A peut destituer un membre du CE-A sur raison valable. Cela nécessite un vote accepté par les deux tiers des membres du CE-A. Le membre visé n’a pas le droit de vote sur la proposition. La proposition devra être entérinée par l’AG. 
\item Les décisions doivent se prendre à main levée et exigent une majorité absolue des votes des membres du CE-A pour entériner une proposition.
\item Tous les membres du CE-A se doivent d’être présents à l’AG normale de la promotion
\end{itemize}

\subsubsection*{Section 6.01 : Les décisions impliquant l'autre alternance}
\addcontentsline{toc}{subsubsection}{Section 6.01 : Les décisions impliquant l'autre alternance}
Si une décision implique clairement l’autre alternance, la présidence du CE-A doit avertir la présidence du CE-P pour qu’il consulte les membres du CE-P. \\

Si un membre du CE-P considère qu’une décision prise par le CE-A et pour laquelle il n’a pas été consulté est dommageable à l’alternance qu’il représente, il doit alors en informer sa présidence, qui communiquera avec la présidence du CE-A s’il juge aussi cette situation problématique. La décision est alors suspendue jusqu’à ce que le CE-P ait pris un vote sur cette proposition.\\

S’il advenait que le CE-A et le CE-P ne soient pas d’accord sur l’application du présent règlement, la décision du CE-P prédomine et la proposition est mise en dépôt. Le litige peut cependant être apporté en AG par la présidence du CE-A et avec un vote à majorité renforcée au 2/3, la décision du CE-P sera infirmée.

\subsection*{Article VII. La présence aux réunions}
\addcontentsline{toc}{subsection}{Article VII. La présence aux réunions}
Un membre du CE se doit d‘être présent aux réunions; la date, l’heure et le lieu de la tenue des réunions doivent donc être pris avec l’intention de maximiser la participation. Cependant, un membre élu qui est absent à plus de deux (2) réunions de façon consécutive sans raison valable, ou à plus de quatre (4) réunions durant la session se voit démis de ses fonctions immédiatement après la constatation de ce fait par le conseil. Le poste est alors déclaré vacant. Un membre qui assisterait à une réunion au moyen d’un équipement audio lui permettant d’interagir activement avec les autres membres du comité est reconnu présent à la réunion : par exemple, une réunion téléphonique, une réunion par webcam ou tout autre moyen permettant une interaction immédiate et continue.

\subsection{Article VIII. Démission d’un poste}
\addcontentsline{toc}{subsection}{Article VIII.	Démission d’un poste}
Si un membre démissionne d’un poste, celui-ci doit faire parvenir sa lettre de démission au CEA dans les plus brefs délais. La démission devient effective une fois la lettre reçue par le CEA. Le CEA peut nommer une personne pour assurer l’intérim du poste, s’il le juge nécessaire. Le CEA doit publiciser, à l’aide des moyens de communication usuels (Facebook, courriels, AGEGnews), la vacance du poste en question. La rétribution se fera tel que prévenu à l’article VII du règlement 6 du présent document. L’AG entérine les changements encourus.

\subsection{Article IX. Destitution d’un membre du CEA}
\addcontentsline{toc}{subsection}{Article IX. Destitution d’un membre du CEA}
Le CEA peut, pour une raison jugée valable, destituer un membre du CEA, tel que prévu à l’article VI du présent règlement. La destitution doit se faire par un vote au 2/3 des membres du CEA, présents ou absents lors de la prise de décision. Le membre en question ne peut voter sur la proposition. Le membre démis de ses fonctions doit remettre tous les documents, objets, codes nécessaires au bon fonctionnement des activités de la promotion. La destitution est effective immédiatement une fois le vote effectué par le CEA.
