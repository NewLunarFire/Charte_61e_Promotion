
\section*{Règlement 7 - Relatif aux dispositions financières} 
\addcontentsline{toc}{section}{Règlement 7 - Relatif aux dispositions financières}

\subsection*{Article I. L'exercice financier}
\addcontentsline{toc}{subsection}{Article I. L'exercice financier}
L'exercice financier de la promotion se termine le 31 décembre de chaque année. 

\subsection*{Article II. Formulaires de dépôt de facture et d'argent}
\addcontentsline{toc}{subsection}{Article II. Formulaires de dépôt de facture}
Tous les formulaires de dépôt de facture et d'argent de la promotion devront être signés par deux (2) des trois (3) signataires, qui sont la trésorerie, la présidence et la vice-présidence du CE-A. Tout chèque payable à la promotion doit être remis à la coordonnatrice administrative par la trésorerie de la promotion. Le formulaire de dépôt doit être signé par deux des trois « signataires ». Les chèques doivent être faits à l'ordre de « AGEG - Promotion 61 ».

\subsection*{Article III. Les contrats}
\addcontentsline{toc}{subsection}{Article III. Les contrats}
Les contrats et autres documents requérant la signature de la promotion seront au préalable approuvés par le CE-A et sur telle approbation, seront signés par deux (2) des trois (3) signataires de la promotion. Les contrats doivent être vus et révisés par le VPAL de l'AGEG, pour ensuite être approuvés par le CA de l'AGEG.

\subsection*{Article IV. Inventaire}
\addcontentsline{toc}{subsection}{Article IV. Inventaire}
Faire l’inventaire des avoirs de la promotion 61 avant le 31 décembre de chaque année. La direction 5@8 est responsable de transmettre l’inventaire des avoirs de la promotion en ce qui concerne les 5@8. La diraction Affaires sociales est responsable de faire et de transmettre l’inventaire des avoirs de la promotion en ce qui concerne les objets promotionnels. Ces inventaires doivent être remis à la date prévue par la coordonnatrice administrative.

\subsection*{Article V. Le transfert des avoirs d'une session à l'autre}
\addcontentsline{toc}{subsection}{Article V. Le transfert des avoirs d'une session à l'autre}
Au début de chaque session, la gestion des avoirs de la promotion sera transférée au nouveau CE-A. Les nouveaux signataires doivent être nommés à la coordonnatrice administrative pour avoir un minimum de deux signataires. Le capital propre de chaque session ne doit en aucun cas être en dessous de 50 \$, à voir ici que le CE-A ne peut pas dépenser plus d’argent qu’il n’en aura accumulé au cours d’une session courante. Advenant une dépense importante et nécessaire qui devrait transgresser cette règle, le CE-A devra au préalable obtenir l’autorisation du CE-P avant d’effectuer ladite dépense.

\subsection*{Article VI. Déplacement d'un membre dans le cadre de ses fonctions}
\addcontentsline{toc}{subsection}{Article VI. Déplacement d'un membre dans le cadre de ses fonctions}
Advenant le cas où un membre du CE-A ait à se déplacer pour effectuer une tâche au nom de la promotion et qu’il ait à utiliser sa voiture, ses déplacements seront alors payés en échange de la valeur du kilométrage (0.25 \$/km) effectué et de la facture de stationnement, s’il y a lieu. Ses déplacements doivent être approuvés par les autres membres du CE-A. Si le membre d’un comité doit se déplacer en voiture dans le cadre de ses fonctions, les dépenses liées au déplacement doivent être approuvées par la direction du comité et celles-ci doivent être présentées au CE-A sur une base mensuelle avant d’être remboursées.

\subsection*{Article VII. Retrouvailles après les 5 premières années}
\addcontentsline{toc}{subsection}{Article VII. Retrouvailles après les 5 premières années}
Afin d'assurer les premières retrouvailles, un montant minimum de trois mille (3 000)\$ sera mis de côté et gardé dans le compte en banque de l’AGEG. La présidence à vie de la promotion (présidence à l’automne 2019) sera responsable de l'organisation de ces retrouvailles avec la possible collaboration de l'AGEG. La tenue des retrouvailles se fera selon la décision de la présidence.

\subsection*{Article  VIII.  Article vestimentaire du comité 5@8.}
\addcontentsline{toc}{subsection}{Article  VIII.  Article vestimentaire du comité 5@8.}
Un montant de 20.00\$ par membre du comité est réservé  pour  le  comité  5@8  afin  de  financer l’achat d’un article vestimentaire permettant leur identification durant les 5@8. Leur nom ou surnom ainsi que leur poste doit être apposé clairement et visiblement sur l’article.