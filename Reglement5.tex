\section*{Règlement 5 - Relatif à la définition des tâches} 
\addcontentsline{toc}{section}{Règlement 5 - Relatif à la définition des tâches}

\subsection*{Article I. La présidence}
\addcontentsline{toc}{subsection}{Article I. La présidence}
La présidence est l'officier exécutif en chef de la promotion. Il doit
\begin{itemize}
    \item Préparer et convoquer toutes les assemblées générales et les réunions du CE;
    \item Préparer, avec le secrétariat, les ordres du jour des réunions du conseil sur lesquels il siège;
    \item Préparer, avec son conseil exécutif, l’ordre du jour des assemblées générales ordinaires;
    \item Superviser l'exécution des décisions de l'AG;
    \item Remplir tous les devoirs inhérents à sa charge et exercer tous les pouvoirs qui pourront lui être attribués par l'AG
    \item Être signataire de la promotion;
    \item Voir à l'application du présent règlement;
    \item Être membre du conseil d’administration de l’AGEG lors qu’il est sur le CE-A, à moins de désigner un autre membre du CE pour prendre sa place;
    \item Créer le dossier de partenariat et coordonner le recherches de commanditaires ainsi que la publicité pour ceux-ci;
    \item Assurer la transition avec le CE-P;
\end{itemize}

La dernière présidence de la promotion est responsable à vie des actifs et des passifs de la 61e promotion de génie ainsi que de l'organisation des premières retrouvailles (compte en banque, factures à payer, etc.).

\subsection*{Article II. La vice-présidence}
\addcontentsline{toc}{subsection}{Article II. La vice-présidence}

La vice-présidence est adjointe à la présidence. Elle doit:
\begin{itemize}
    \item Assurer les pouvoirs et les responsabilités de la présidence en son absence;
    \item S'assurer de l'application de la charte et de ses règlements;
    \item Superviser l'avancement du travail fait par les différents comités;
    \item Vérifier et discuter de l’avancement de leurs tâches des différentes directions et le documenter dans un fichier de suivi;
    \item Être signataire de la promotion;
    \item Être membre du comité 5@8;
    \item Distribuer les Points Génie à ceux qui travaillent lors des activités de financement de la promotion;
    \item Embaucher et superviser le personnel du 5@8;
    \item Arbitrer les litiges concernant les Points Génie, l'application de ce règlement et l'interprétation de tous les règlements qui viennent s'ajouter aux règlements généraux;
    \item Fixer toute rémunération non réglementée par l'article IV du règlement 8;
    \item Compiler et tenir à jour le compte des Points Génie des membres, l'avoir du compte personnel ainsi que le plateau financier personnel des membres au minimum une fois par semaine sur le site internet de la promotion;
    \item Sortir un rapport de la comptabilité des comptes personnels de chaque membre à la fin de chaque session et l’enregistrer dans le nuage de l’AGEG afin d'éviter toutes pertes ou falsifications des données lors d'un changement du CE-A.
\end{itemize}

\subsection*{Article III. La trésorerie}
\addcontentsline{toc}{subsection}{Article III. La trésorerie}
\begin{itemize}
    \item Être signataire de la promotion;
    \item Rédiger le budget de la promotion sur ordinateur avec une copie enregistrée dans les archives de la promotion;
    \item Mettre le budget à jour à chaque semaine;
    \item Tenir un relevé précis des biens, des dettes, des recettes et des déboursés de la promotion dans le fichier approprié à cette fin;
    \item Remettre toutes les factures à la coordonnatrice administrative et conserver toutes les copies de factures relatives à la promotion dans un lieu approprié;
    \item Déposer, auprès de la coordonnatrice administrative, les actifs déposables de la promotion;
    \item Préparer les planifications budgétaires et les relevés des dépenses;
    \item Recevoir les états financiers et les bilans financiers de la coordonnatrice administrative;
    \item Présenter en AG le bilan de la session précédente et de la session en cours à la fin de la session de son mandat;
    \item Former la prochaine trésorerie et s’assurer qu'il a la compréhension des outils dont la promotion dispose pour effectuer le suivi budgétaire, présenter le bilan et fournir une trace de contrôle du flux de trésorerie;
    \item S'assurer de suivre la formation sur la gestion du budget des groupes et des promotions;
    \item remplir son budget selon les normes établies par le guide de gestion du budget des groupes, relatif au règlement 50 de l'AGEG;
    \item Être membre du comité 5@8;
    \item Compter l’argent à la fin de tous les 5@8 avec la sous-direction petite caisse;
    \item Déposer l’argent en dépôt de nuit le jeudi avec un autre membre du comité 5@8 que le sous-direction petite caisse.
\end{itemize}
 
 De plus, la trésorerie de l’automne 2018 doit s’assurer de faire un suivi des derniers comptes à recevoir à l’hiver avant de payer les Points Génie.
 

\subsection*{Article IV. Le secrétariat}
\addcontentsline{toc}{subsection}{Article IV. Le secrétariat}
\begin{itemize}
    \item Assister à toutes les réunions de son CE et en rédiger les procès-verbaux;
    \item Aider la présidence dans sa rédaction de l'ordre du jour et communiquer aux membres de la promotion les rapports de comités, les propositions et autres documents officiels;
    \item Rédiger tous les papiers et lettres requis par la promotion;
    \item Tenir un dossier contenant tous ces documents et le rendre disponible sur le nuage de l’AGEG;
    \item Doit distribuer, par le moyen approprié, la correspondance interne et externe du CE;
    \item Assurer l’application de l’article VIII du règlement 3 relatif à la présence aux réunions;
    \item Corriger les procès-verbaux dans un délai d'une semaine et les rendre disponibles sur le nuage de la promotion.
\end{itemize}

\subsection*{Article V. Le webmestre}
\addcontentsline{toc}{subsection}{Article V. Le webmestre}
\begin{itemize}
    \item Tenir à jour la page web et le nuage de la promotion en y mettant toutes les informations utiles aux membres de la promotion;
    \item Annoncer toutes les activités de la promotion sur la page web;
    \item Agir comme modérateur pour le forum des membres;
    \item Préparer un système de gestion des Points Génie et de vote en ligne;
    \item Concevoir et mettre en place le système d'inscription des membres pour le bal et le voyage;
    \item Réaliser toute amélioration au site web jugée pertinente par le CE-A.
\end{itemize}

\subsection*{Article VI. Les représentants}
\addcontentsline{toc}{subsection}{Article VI. Les représentants}
\begin{itemize}
    \item Acheminer l’information aux membres de leurs programmes respectifs;
    \item Prendre position et représenter leur programme lors des débats au conseil exécutif;
    \item Exécuter les tâches nécessaires pour l’exécution de leur mandat.
\end{itemize}

\subsection*{Article VII. Direction}
\addcontentsline{toc}{subsection}{Article VII. Direction}
Les descriptions des postes des différentes directions de comités ainsi que de leur comité sont définies ci-bas. Les directions ont la responsabilité d'accomplir les tâches inhérentes aux activités pour lesquelles ils ont été élus. Si la somme de travail est trop importante ou si l’ensemble des poste d’un comité n’est pas comblé en AG, il incombe à la direction de former un comité. Les directions ont également la responsabilité de distribuer les Points Génie au sein de leur comité. Tous les postes à combler dans un comité devront être affichés. Dans un comité, la direction devra favoriser au maximum la diversité des programmes. En cas de litige, le comité des Points Génie pourra intervenir. Ils doivent rédiger un rapport d’étape à la fin de leur mandat. 

\subsubsection*{Section 7.01 : Directions 5@8}
\addcontentsline{toc}{subsubsection}{Section 7.01 : Directions 5@8}
\begin{itemize}
    \item Terminer la formation de son comité suite à son élection si des postes sont laissés vacants tout en respectant les modalités décrites à l'article 5 du règlement 4;
    \item Préparer et animer les réunions hebdomadaires du comité 5@8 et rédiger les comptes-rendus;
    \item Veiller à l'organisation et être présent à tous les 5@8;
    \item Participer à la bonne entente entre la sécurité et la Faculté (comité OSS);
    \item Donner le support nécessaire à tous ses sous-directions;
    \item Respecter l'entente avec le G3 concernant les 5@8 communs;
    \item Sièger sur le conseil d'administration de l'AGEG pour la session en cours ainsi que sur la commission des affaires campus de la FEUS.
    \item S’occuper du renouvellement du permis d’alcool avec la sécurité;
\end{itemize}

\subsubsection*{Section 7.02 : Direction Affaires sociales}
\addcontentsline{toc}{subsubsection}{Section 7.02 : Direction Affaires sociales}
\begin{itemize}
    \item Organiser tous les partys de la promotion finissante;
    \item Être responsable de la semaine de financement;
    
    \item Commander les articles promotionnels;
    \item Maintenir de bonnes relations avec les autres promotions et s'assurer que l'entente interpromo est respectée;
    \item Assurer la coordination avec la vice-présidence aux affaires sociales de l'AGEG;
    \item Être responsable des inventaires de stock de la promotion omit ceux du comité 5@8.
\end{itemize}

\subsubsection*{Section 7.03 : Direction Bal/Voyage}
\addcontentsline{toc}{subsubsection}{Section 7.03 : Direction Bal/Voyage}
\begin{itemize}
    \item Organiser le bal des finissants en sélectionnant l'entité responsable de l'organisation de celui-ci;
    \item Organiser le voyage de fin de baccalauréat de la promotion;
    \item Travailler avec le VPAL de l'AGEG pour la rédaction et l'approbation des contrats;
    \item Présenter les contrats qu'il désire signer au CA l'AGEG.
\end{itemize}

\subsubsection*{Section 7.04 : Direction Casino/Jonc}
\addcontentsline{toc}{subsubsection}{Section 7.04 : Direction Casino/Jonc}
\begin{itemize}
    \item Organiser les activités de la remise du jonc et casino de sa promotion et assurer le bon fonctionnement de celle-ci;
    \item Commander les joncs;
    \item Réserver la salle pour la cérémonie de remise de sa promotion;
    \item Embaucher le personnel du casino et de la cérémonie jonc pour les activités de la promotion précédente;
    \item Assurer le bon déroulement de la soirée casino.
\end{itemize}

\subsubsection*{Section 7.05 : Direction Souvenirs}
\addcontentsline{toc}{subsubsection}{Section 7.05 : Direction Souvenirs}
\begin{itemize}
    \item Trouver le photographe qui prendra les photos de tous les finissants et qui réalisera la mosaïque;
    \item Organiser les séances de photographies sur 3 sessions (hiver, été et automne 2018) et de s’assurer que tous les finissants ont pu être posés;
    \item Choisir une compagnie d’impression pour la production de l’album des finissants;
    \item Est responsable du montage en entier de l’album des finissants en collaboration avec son comité;
    \item Amasser tous les textes et photos souvenirs d’évènements et d’activités marquantes qui y figureront;
    \item Produire la murale de la promotion dans les tunnels et l’oeuvre d’art du 5e étage de la Faculté de Génie;
    \item Travailler avec le VPAL de l’AGEG pour la rédaction et l’approbation des contrats, puis les présenter au CA de l’AGEG.
\end{itemize}

\subsubsection*{Section 7.06 : Direction Journée Carrière}
\addcontentsline{toc}{subsubsection}{Section 7.06 : Direction Journée Carrière}
\begin{itemize}
    \item Assister aux réunions du Service des stages et du placement (SSP);
    \item Organiser la journée carrière de l’automne 2018 durant l’été et l’automne 2018;
    \item Recruter des étudiants de la 61\textsuperscript{e} promotion pour aider la veille et la journée même (rémunération en Points Génie);
    \item Assurer de la présence de plusieurs entreprises à la journée carrière.
\end{itemize}

\subsection*{Article VIII. Rapport d'activité}
\addcontentsline{toc}{subsection}{Article VIII. Rapport d'activité}
Chaque activité (5@8, casino, etc.) devra faire l'objet d'un rapport. Le responsable de l'activité verra à la production du rapport. Celui-ci comprendra le budget détaillé de l'activité, les Points Génie accordés ainsi que le nombre d'heures données à chaque personne impliquée. Le rapport devra comprendre aussi le taux de participation et toutes autres remarques pertinentes. Les signatures des personnes responsables et la date devront apparaître à la fin du rapport. Ces rapports devront être conservés dans les archives de la promotion du CE-A. Le rapport d’activité comprenant les Points Génie et leur gestion ne sera fait que lorsque la promotion sera finissante. Ces rapports d’activité serviront également à rédiger les rapports d’étape des membres du CE-A. La date limite est fixée à sept jours après la fin de l’activité. Ce rapport est remis au CE-A qui le placera dans les archives de la promotion. Le responsable de l’activité qui ne dépose pas de rapport d’activité avant la date limite se verra soustraire deux (2) Points Génie.

\subsection*{Article IX. Rapport d'étape}
\addcontentsline{toc}{subsection}{Article IX. Rapport d'étape}
Afin d'assurer un suivi constant des activités de la promotion, tous les membres du CE-A auront comme tâche de remettre un rapport d'étape à la fin de leur mandat. Le secrétariat et les représentants de programme font exception à cette règle. Si un membre du CE ne produit pas un rapport d’étape à la fin de son mandat, il se verra soustraire cinq (5) Points Génie. La date limite est fixée à cinq (5) jours après le début du mandat du nouveau CE-A; ce rapport est remis au CE-A sur le nuage de la promotion.  

\subsection*{Article X. Support au déroulement de l'activité 5@8 prolongée}
\addcontentsline{toc}{subsection}{Article X. Support au déroulement de l'activité 5@8 prolongée}
Les membres du CE-A doivent être disponibles afin d’assister le comité 5@8 lors du déroulement de l’activité prolongée qui suit les trois (3) premières heures de l’activité sociale usuelle. Les membres du CE-A ne recevront aucun Point Génie supplémentaire pour ce travail. Toutefois, s’ils ne sont pas disponibles et qu'il faut trouver un remplaçant, les Points Génie devant être attribués pour trouver un remplaçant seront retranchés des Points Génie du membre du CE-A. Dans le cas d’une absence motivée ou d’un cas de force majeure, si la raison est approuvée à majorité renforcée au 2/3 par le CEA, les points ne seront pas retranchés du membre du CE-A. Les membres du CE-A doivent supporter le comité 5@8 pour une (1) activité prolongée de la FEUS et jusqu'à deux (2) organisées par la promotion. 