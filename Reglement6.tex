
\section*{Règlement 6 - Relatif à l'organisation et la tenue des 5@8} 
\addcontentsline{toc}{section}{Règlement 6 - Relatif à l'organisation et la tenue des 5@8}


\subsection*{Article I. Définition d'une activité 5@8}
\addcontentsline{toc}{subsection}{Article I. Définition d'une activité 5@8}
L’activité 5@8 est une activité généralement tenue au Salon de la Faculté de Génie le jeudi soir. L’activité débute à dix-sept (17) heures et se termine à vingt (20) heures ou à une autre heure précisée par l’organisation. L’activité est organisée par les membres du comité 5@8 et est coordonnée par les directions 5@8.

\subsection*{Article II. Définition de l'activité 5@8 prolongée}
\addcontentsline{toc}{subsection}{Article II. Définition de l'activité 5@8 prolongée}
L’activité 5@8 prolongée est une activité tenue une à deux fois par session généralement à la suite d’un 5@8. L’activité est organisée et coordonnée par les membres du comité 5@8 et des membres du CE-A. 

\subsection*{Article III. Formation et composition du comité 5@8}
\addcontentsline{toc}{subsection}{Article III. Formation et composition du comité 5@8}
Le comité 5@8 sera formé par l’AG selon un processus favorisant la diversité des concentrations au sein du comité. Le comité est composé au minimum des membres suivants :
\begin{itemize}
\item Les 2 directions 5@8 du CE-A;
\item La vice-présidence du CE-A;
\item 2 sous-directions bière;
\item 1 sous-direction caisse;
\item 2 sous-directionss fort;
\item 1 sous-direction repas; 
\item 1 sous-direction radio;
\item 1 sous-direction animation;
\item 1 sous-direction sécurité;
\item 1 sous-direction photographie;
\item La trésorerie du CE-A.
\end{itemize}

\subsection*{Article IV. Obligation des membres du comité 5@8}
\addcontentsline{toc}{subsection}{Article IV. Obligation des membres du comité 5@8}
Chaque membre du comité 5@8 se doit d’être présent à toutes les réunions du comité 5@8 et à tous les 5@8 afin d’aider au bon fonctionnement de ces activités sociales. Cependant, un membre élu qui est absent à plus de deux (2) réunions de façon consécutive sans raison valable, ou à plus de quatre (4) réunions durant la session, se voit démis de ses fonctions immédiatement après la constatation de ce fait par le comité 5@8. Ceci inclut la préparation du 5@8, ainsi que la fermeture du 5@8. Le comité est également présent à d’autres évènements dans la faculté où il y a distribution de boissons alcoolisées par exemple la soirée casino et les retrouvailles.

\subsection*{Article V. Rôles des membres du comité 5@8}
\addcontentsline{toc}{subsection}{Article V. Rôles des membres du comité 5@8}

\subsubsection*{Section 5.01 : Directions 5@8}
\addcontentsline{toc}{subsubsection}{Section 5.01 : Directions 5@8}
Se référer à la 7.01 du règlement 5 pour la description de tâches des directions 5@8.

\subsubsection*{Section 5.02 : Sous-directions bière}
\addcontentsline{toc}{subsubsection}{Section 5.02 : Sous-directions bière}
\begin{itemize}
\item Prendre connaissance du contrat de bière de l’AGEG;
\item Effectuer la commande bière en début de semaine et assurer le lien avec le fournisseur de bière de l’AGEG;
\item Remplir les réfrigérateurs;
\item Préparer le bar avant tous les 5@8;
\item S’assurer du bon déroulement de la vente de bière lors de tous les 5@8;
\item Réaliser et gèrer l’inventaire des stocks de bière.
\end{itemize}

\subsubsection*{Section 5.03 : Sous-direction fort}
\addcontentsline{toc}{subsubsection}{Section 5.03 : Sous-direction fort}
\begin{itemize}
\item S’assurer d’avoir les stocks suffisants pour le bar à cocktails (alcool et jus) et procéder aux achats si nécessaire;
\item Préparer le bar à cocktails avant tous les 5@8;
\item S’assurer du bon déroulement de la vente de cocktails lors de tous les 5@8;
\item Réaliser et gèrer l’inventaire des stocks de fort/boissons.
\end{itemize}

\subsubsection*{Section 5.04 : Sous-direction caisse}
\addcontentsline{toc}{subsubsection}{Section 5.04 : Sous-direction caisse}
\begin{itemize}
\item S'occuper de l'argent lors des 5@8;
\item Vider les caisses des différents postes (bouffe, bar à fort, bar à bière, etc.);
\item Compter l'argent amassé après chaque 5@8;
\item Assister la trésorerie du CE-A lors des transferts d’argent en espèce.
\end{itemize}

\subsubsection*{Section 5.05 : Sous-direction repas}
\addcontentsline{toc}{subsubsection}{Section 5.05 : Sous-direction repas}
\begin{itemize}
\item Préparer la disposition des tables pour la vente de repas;
\item Recevoir et payer les commandes lors de tous les 5@8;
\item S’assurer du bon déroulement de la vente de repas lors de tous les 5@8;
\item S’assurer des stocks nécessaires au bon fonctionnement du lave-vaisselle et procéder aux achats nécessaires;
\item Réaliser et gérer l’inventaire des stocks de nourriture.
\end{itemize}

\subsubsection*{Section 5.06 : Sous-direction radio}
\addcontentsline{toc}{subsubsection}{Section 5.06 : Sous-direction radio}
\begin{itemize}
\item Préparer la disposition des consoles et boîtes de sons lors de tous les 5@8;
\item S’assurer d’avoir le matériel nécessaire pour la diffusion de la musique;
\item Coordonner la musique avec les activités lors de tous les 5@8;
\item Être le responsable de la radio auprès de l’AGEG.
\end{itemize}

\subsubsection*{Section 5.07 : Sous-direction animation}
\addcontentsline{toc}{subsubsection}{Section 5.07 : Sous-direction animation}
\begin{itemize}
\item Animer activement tous les 5@8 entre autres avec le jeu de la semaine;
\item Assurer la sélection de thèmes et favoriser le déguisement du comité;
\item Publiciser les 5@8 par les moyens jugés efficaces (Affiches, Facebook, télévisions Bouche-à-oreille...);
\item Assister les autres sous-directions durant les activités du comité 5@8;
\item Assurer la sécurité aux autobus à compter de 10 minutes avant la fin de l’activité et s’annoncer à la sécurité;
\item Replacer les divans et les tables de jeu dans le salon de l’AGEG le lendemain matin des 5@8.
\end{itemize}

\subsubsection*{Section 5.08 : Sous-direction sécurité}
\addcontentsline{toc}{subsubsection}{Section 5.08 : Sous-direction sécurité}
\begin{itemize}
\item S’assurer de la sécurité des lieux, des équipements et des participants lors des 5@8, des activités 5@8 prolongées et des autres évènements de promotion applicables;
\item S’occuper de remettre à l’ordre les lieux après les 5@8;
\item Gérer les quarts de travail de la sécurité pendant les 5@8;
\item Réaliser une analyse de risques pour les 5@8;
\item Réaliser un plan d’intervention d’urgence;
\item Délimiter le périmètre du 5@8 avant l’événement.
\end{itemize}

\subsubsection*{Section 5.09 : Sous-direction photographe}
\addcontentsline{toc}{subsubsection}{Section 5.09 : Sous-direction photographe}
\begin{itemize}
\item Prendre des photos à tous les 5@8 de la session;
\item Effectuer le tri des photos pour enlever les photos compromettantes et les faire accepter par les directions 5@8;
\item Publier les photos avant le 5@8 de la semaine suivante sauf pour raison majeure jugée acceptable par le comité 5@8.
\end{itemize}

\subsection*{Article VI. L'affichage des postes (bénévoles vente, bière et sécurité)}
\addcontentsline{toc}{subsection}{Article VI. L'affichage des postes (bénévoles vente, bière et sécurité)}
La création de l’horaire de travail est la responsabilité de la vice-présidence. Ce dernier se doit d’offrir aux membres une période d’affichage d’au moins quarante-huit (48) heures, de respecter la procédure de priorité des Points Génie et de faire parvenir l’horaire de travail au moins vingt-quatre (24) heures à l’avance. Si la météo est incertaine, l'horaire des personnes « en renfort » sera confirmé à 12 h la journée de l'événement. \\

Relatif au 5@8, les plages de disponibilités se divisent en bloc de trois (3) heures. Les distributions des heures de travail sont faites par bloc d’une (1) heure à la discrétion de la sous-direction concernée. La vice-présidence peut cependant ouvrir une plage horaire supplémentaire pour accommoder les membres en stage.\\

La procédure suivante s'applique pour déterminer les priorités :
\begin{itemize}
\item Les tâches seront attribuées aux membres ayant postulé et possédant le moins de Points Génie accumulés;
\item Dans le cas d'égalité des Points Génie, un membre régulier actif sera priorisé, suivi par un membre régulier passif puis un membre non-régulier;
\item Ensuite, la règle du premier arrivé, premier servi s'appliquera. 
\end{itemize}

Advenant une surévaluation du nombre de bénévoles nécessaire ou un faible achalandage lors d’une activité de financement, le comité 5@8 se réserve le droit de relever de leurs fonctions les bénévoles jugés superflus par le sous-direction en question. Le bénévole qui est relevé de ses fonctions sera celui au sein du poste ayant le plus de Points Génie accumulés. De plus, la rémunération en point génie de ce bénévole sera équivalente au nombre d’heures travaillées arrondies à l'entier supérieur, avec une valeur minimale de 1.\\

Lorsqu’ils postulent, les membres se doivent de tenir leur engagement ainsi que de respecter les consignes et conditions de travail relatives à l’évènement.

\subsubsection*{Section 6.01 : Sanctions}
\addcontentsline{toc}{subsubsection}{Section 6.01 : Sanctions}
Tout membre qui commet une action mettant en péril la réputation, le bon fonctionnement ou la pérennité des 5@8 se verra attribué une sanction qui est répertoriée, ou non, dans le Tableau 2 suivant. La faute doit être recconu par les directions 5@8 et la sous-direction du départment où le bénévole travaillait. La sanction doit ensuite être déterminer par le commité de sanction (les deux directions 5@8, la présidence et la vice-présidence).  Le degré de la sanction doit être proportionnel à la faute commise et à l’historique du fautif. La collusion, l’usurpation d’identité, la vente d’alcool passée les heures permises sont des exemples d’actions mettant en péril la réputation, le bon fonctionnement et la pérennité des 5@8. Une fois la sanction déterminée, l’état de la situation doit être rapportée au CEA par la vice-présidence ou par la directions 5@8.\\

\begin{table}[H]
\centering
\caption{Sanctions en fonction de la gravité}
\label{Sanction2}
\begin{tabular}{|l|l|}
\hline
\textbf{Degré} & \textbf{Sanctions} \\
\hline
1 & \begin{tabular}[c]{@{}l@{}}Le membre reçoit un avertissement\\ verbal\end{tabular} \\ \hline
2 & \begin{tabular}[c]{@{}l@{}}Les Points Génie attribués au membre\\ pour les heures où les fautes ont été \\ commises lui sont retirés.\end{tabular} \\ \hline
3 & \begin{tabular}[c]{@{}l@{}}Il est interdit au  membre de travailler \\ aux 5@8 pour le reste de la session et ses \\ Points Génie lui sont retirés.\end{tabular} \\ \hline
4 & \begin{tabular}[c]{@{}l@{}}Le membre est banni de l'ensemble des \\ 5@8 en tant que bénévole et ses Points\\  Génie lui sont retirés.\end{tabular} \\ \hline
\end{tabular}
\end{table}