
\section*{Règlement 8 - Relatif aux Points Génie} 
\addcontentsline{toc}{section}{Règlement 8 - Relatif aux Points Génie}

\textbf{NOTE 1 : Ce règlement entre en vigueur à l'automne 2018, lors de la passation des pouvoirs de la 60$^e$ promotion à la 61$^e$ promotion.}

\subsection*{Article I. Le but du règlement}
\addcontentsline{toc}{subsection}{Article I. Le but du règlement}
Le but du présent règlement est d'établir les normes selon lesquelles l'argent accumulé par la Promotion sera redistribué à chacun de ses Membres afin d’alléger le coût des activités finales du baccalauréat.

\subsection*{Article II. Définition des Points Génies}
\addcontentsline{toc}{subsection}{Article II. Définition des Points Génies}
Les Points Génie sont une unité fictive distribuée aux Membres en retour de leur contribution aux activités de financement de la Promotion. La valeur ces Points Génie est déterminée selon l’article suivant : Art. 3.2.a.

\subsubsection*{Section 2.01 : Contenu du compte commum de la promotion}
\addcontentsline{toc}{subsubsection}{Section 2.01 : Contenu du compte commum de la promotion}
Le compte commun des Membres contient :
\begin{itemize}
\item Les profits de toutes les activités organisées par la Promotion pour ramasser des fonds;
\item Un minimum de quatre-vingts pour cent (80\%) des commandites recueillies par les membres.
\end{itemize}

\subsubsection*{Section 2.02 : Répartition du contenu du compte commum des membres}
\addcontentsline{toc}{subsubsection}{Section 2.02 : Répartition du contenu du compte commum des membres}
Seuls les membres peuvent profiter de la distribution de l’argent de ce compte.

\textbf{(a) Banque des Points Génie}
La totalité du compte commun sera divisée entre chacun des membres selon le nombre de Points Génie que les membres auront accumulés, jusqu’à concurrence de quatre-vingts (80) Points Génie. Au courant de l’année, la limite de quatre-vingts (80) Points Génie sera recalculée en fonction des prévisions du moment et devra être approuvée en AG. Un membre n’ayant pas fait de Points Génie ne pourra bénéficier de la banque de Points Génie. Le taux du Point Génie sera déterminé de la façon itérative suivante :

$$ V_{PG} (\text{1ère itération}) = \frac{V_{CC} - V_R}{T_{PG}} $$

\begin{description}
    \item $V_{PG}$ : Valeur d'un Point Génie
    \item $V_{CC}$ : Valeur du compte commun
    \item $V_{R}$ : Valeur de l'argent mis de côté pour les retrouvailles (environ 5000\$)
    \item $T_{PG}$ : Total de Points Génie accumulés par l'ensemble des membres
\end{description}

Ensuite, l’argent nécessaire à débourser toutes les dépenses des membres ayant amassé suffisamment de Points Génie sera déduit du compte commun et la valeur du Point Génie sera recalculée.

$$ V_{PG} (\text{2e itération}) = \frac{V_{CC}'}{T_{PG}'} $$

\begin{description}
    \item $V_{PG}$ : Valeur d'un Point Génie
    \item $V_{CC}'$ : Valeur du compte commun après la première itération
    \item $T_{PG}'$ : Total de Points Génie accumulés par l'ensemble des membres restants
\end{description}

Il en va ainsi jusqu’à l’obtention d’une valeur du Point Génie ne permettant pas de débourser la totalité des dépenses d’un des membres restants. Le taux du Point Génie ne pourra cependant être évalué que dans le dernier mois du baccalauréat, soit au moment où toutes les dépenses auront été calculées. Dans le cas où un membre ne fait plus partie de la promotion, les Points Génie qu’il aura accumulés seront enlevés du nombre total des Points Génie de l’ensemble des membres.

\subsection*{Article III. Compte personnel des membres de la promotion}
\addcontentsline{toc}{subsection}{Article III. Compte personnel des membres de la promotion}

\subsubsection*{Section 3.01 : Définition du compte personnel des membres}
\addcontentsline{toc}{subsubsection}{Section 3.01 : Définition du compte personnel des membres}
Un compte personnel est le compte fictif géré par la vice-présidence permettant d’accumuler les sommes qui seront accumulées au cours de la période pendant laquelle la promotion est finissante.

\subsubsection*{Section 3.02 : Contenu du compte personnel}
\addcontentsline{toc}{subsubsection}{Section 3.02 : Contenu du compte personnel}
À la fin de chaque session, à un moment jugé opportun par le comité des Points Génie, le contenu du compte personnel de chaque membre sera calculé. Les items suivants sont considérés dans le calcul du contenu du compte personnel du membre :
\begin{itemize}
\item Un maximum de vingt pour cent (20\%) des commandites amassées;
\item Le nombre de Points Génie accumulés multiplié par le taux du Point Génie.
\end{itemize}

\subsubsection*{Section 3.03 : Utilisation du contenu du compte personnel}
\addcontentsline{toc}{subsubsection}{Section 3.03 : Utilisation du contenu du compte personnel}
Le contenu du compte personnel est utilisé pour payer les frais personnels relatifs aux activités de finissants. Les frais payables sont les suivants :

\begin{itemize}
\item Le montant à payer pour les frais de bal du membre, fixé au montant du forfait choisi (le forfait incluant le logement ainsi que le coût de la soirée du bal pour lui-même et une personne accompagnatrice), à la condition que le membre achète son billet de bal et s’il est accompagné, le billet de bal de la personne accompagnatrice;
\item Le montant dépensé pour l’achat de l’album, à la condition que le membre achète l’album;
\item Le montant dépensé pour l’achat du jonc d’ingénieur, à la condition que le membre achète son jonc;
\item Le montant dépensé pour l’achat des photos de finissant, à la condition que le membre achète un ou des forfaits de photos;
\item Le montant dépensé pour le voyage de fin de baccalauréat, à la condition que le membre achète le forfait pour le voyage de fin de baccalauréat. Un accompagnateur non-membre est permis au voyage;
\item Lorsque l’ensemble de ces frais a été remboursé, le membre atteint son plateau financier personnel. Le surplus du compte personnel excédant le montant du plateau financier personnel sera remis dans la banque de Points Génie.
\end{itemize}

\subsubsection*{Section 3.04 : Utilisation du compte personnel}
\addcontentsline{toc}{subsubsection}{Section 3.04 : Utilisation du compte personnel}
La somme amassée par le membre dans son compte personnel constitue un rabais sur le paiement final des articles mentionnés plus haut. Le rabais final au membre représente le contenu de compte personnel jusqu’à concurrence de son plateau financier personnel. Ces fonds sont non transférables d’un membre à l’autre.\\

La facture incluant le rabais doit être transmise au membre au plus tard le 15 décembre 2019 et le paiement final doit être effectué avant le 31 décembre 2019.

\subsection*{Article IV. L'accumulation de Points Génie}
\addcontentsline{toc}{subsection}{Article IV. L'accumulation de Points Génie}
L’accumulation de Points Génie s'effectuera à partir de la session d'automne 2018. Un membre de la promotion qui s'implique dans le CE-A obtient des Points Génie selon la répartition faite dans le tableau 2. Le nombre de sous-directions ainsi que les Points Génie associés aux différents comités doivent respecter le tableau 3 ci-dessous, exception faite des sous-directions du comité 5@8 présenté au Règlement 10. Un membre de la promotion, qui s’implique lors des activités de financement de la promotion, obtient des Points Génie selon la répartition faite dans le tableau 4. Tous ses Points Génie seront compilés dès la première semaine de chaque session à partir de l'automne 2018. \\

Pour chaque étudiant à chaque session, il y a une limite de deux (2) implications dans un comité ou une (1) implication dans le CE et une (1) implication dans un comité. Lors de l’occupation de deux (2) postes, la moitié des Points Génie du poste donnant le moins de Points Génie sera attribuée à l’étudiant. Cela n'empêche pas d'être membre du CE ou d'un comité autant de fois que désiré. Les bénévoles sécurité peuvent être exigés de rester jusqu’à une demie-heure supplémentaire. Les bénévoles sécurité seront payés un demi-point génie pour la période supplémentaire.

\begin{table}[H]
    \centering
    \caption{Nombre de Points Génie attribués par session aux membres du CE et au comité 5@8}
    \begin{tabular}{|l|c|c|c|c|}
      \hline
       &\multicolumn{4}{|c|}{\textbf{Sessions}} \\ 
      \hline
       & \textbf{A-18} & \textbf{H-19} & \textbf{E-19} & \textbf{A-19} \\
      \hline
      \textbf{Présidence} & 30 & 35 & 35 & 45 \\
      \hline
      \textbf{Vice-présidence/Direction PG} & 25 & 30 & 30 & 30\\
      \hline
      \textbf{Trésorerie} & 25 & 28 & 28 & 30 \\
      \hline
      \textbf{Secrétariat} & 10 & 10 & 10 & 10 \\
      \hline
      \textbf{Webmester} & 15 & 10 & 10 & 10 \\
      \hline
      \textbf{Représentant} & 5 & 5 & 5 & 5 \\
      \hline
      \textbf{Direction Affaires sociales} & 10 & 10 & 10 & 10 \\
      \hline
      \textbf{Direction Bal/Voyage} & 10 & 20 & 20 & 20 \\
      \hline
      \textbf{Direction Souvenirs} & 10 & 20 & 20 & 20 \\
      \hline
      \textbf{Direction Jonc/Casino} & 10 & 8 & 15 & 20 \\
      \hline
      \textbf{Direction Journée Carrière} & 0 & 0 & \multicolumn{2}{|c|}{20} \\
      \hline
      \textbf{Direction 5@8 (1)} & 25 & 30 & 30 & 25 \\
      \hline
      \textbf{Direction 5@8 (2)} & 25 & 30 & 30 & 25 \\
      \hline
      \textbf{Sous-direction bière (1)} & 14 & 22 & 22 & 16 \\
      \hline
      \textbf{Sous-direction bière (2)} & 14 & 22 & 22 & 16 \\
      \hline
      \textbf{Sous-direction fort (1)} & 14 & 22 & 22 & 16 \\
      \hline
      \textbf{Sous-direction fort (2)} & 14 & 22 & 22 & 16 \\
      \hline
      \textbf{Sous-direction sécurité} & 13 & 20 & 20 & 15 \\
      \hline
      \textbf{Sous-direction radio} & 13 & 20 & 20 & 15 \\
      \hline
      \textbf{Sous-direction photographie} & 13 & 20 & 20 & 15 \\
      \hline
      \textbf{Sous-direction repas} & 11 & 16 & 20 & 15 \\
      \hline
      \textbf{Sous-direction animation} & 12 & 18 & 18 & 14 \\
      \hline
      \textbf{Sous-direction caisse} & 11 & 16 & 16 & 13 \\
      \hline
    \end{tabular}
\end{table}

\begin{table}[H]
    \centering
    \caption{Nombre de sous-directions maximales pouvant être attribués par session à chaque comité}
    \begin{tabular}{|l|c|c|c|c|}
      \hline
       &\multicolumn{4}{|c|}{\textbf{Sessions}} \\ 
      \hline
       & \textbf{A-18} & \textbf{H-19} & \textbf{E-19} & \textbf{A-19} \\
      \hline
      \textbf{Comité Bal/Voyage} & 2 & 2 & 2 & 5 \\
      \hline
      \textbf{Comité Souvenir} & 1 & 2 & 2 & 4 \\
      \hline
      \textbf{Comité Jonc/Casino} & 0 & 0 & 1 & 4 \\
      \hline
      \textbf{Comité Journée Carrière} & 0 & 0 & 1 & 1 \\
      \hline
    \end{tabular}
\end{table}

L’attribution des points génie aux sous-directions excluant le comité 5@8 qui composent les différents comités du tableau 4 est fixée à \textbf{8 points par sous-direction et par session}, et ce, peu importe le nombre de sous-direction sélectionnées.

\begin{table}[H]
    \centering
    \caption{Nombre de Points Génie pouvant être distribués par heure de travail des Membres contribuant aux activités de promotion}
    \begin{tabular}{|l|c|}
      \hline
      & \textbf{Nombre de Points Génie} \\
      & \textbf{attribuables par heure} \\
      & \textbf{de service rendu}\\
      \hline
      \textbf{Vente de bière/fort/nourriture (5@8)} & 1 \\
      \hline
      \textbf{Caisse (5@8)} & 1 \\
      \hline
      \textbf{Sécurité (5@8)} & 1 \\
      \hline
      \textbf{Casino} & 1 \\
      \hline
    \end{tabular}
    \label{comite}
\end{table}

\subsubsection*{Section 4.01 : Points Génie pour le casino de la 60$^e$ promotion de génie de l'Université de Sherbrooke}
\addcontentsline{toc}{subsubsection}{Section 4.01 : Points Génie pour le casino de la 60$^e$ promotion de génie de l'Université de Sherbrooke}
La 61$^e$ promotion de génie s’engage à fournir et à rémunérer à même ses propres Points Génie, le personnel nécessaire à la bonne marche du casino de la 60$^e$ promotion selon les modalités décrites dans “L’entente inter-promotions”. Le nombre de Points Génie est fixé à quatre (4) par membre travaillant au casino. 

\subsection*{Article V. Litiges et participation insuffisante}
\addcontentsline{toc}{subsection}{Article V. Litiges et participation insuffisante}
La direction/présidence est le premier responsable de ses membres de comité. Tout litige doit être soumis par écrit au CE-A. Le CE-A contactera la(les) personne(s) en litige. Un ultimatum sera alors donné par le CE-A afin de rétablir la ou les situations en problèmes. En cas de négligence majeure, la direction ou sous-direction peut se voir retirer ses Points Génie par la vice-présidence. Un membre du CE ou d'un comité ou une direction de comité peut se voir refuser ses Points Génie si sa participation est jugée insuffisante par les autres membres du CE ou de ce comité.

\subsection*{Article VI. Attribution des Points Génie}
\addcontentsline{toc}{subsection}{Article VI. Attribution des Points Génie}
L’attribution des Points Génie d’un membre élu se fait au moment où la personne est sélectionnée pour le poste et qu’elle y consent.

\subsection*{Article VII. Démission d'un poste}
\addcontentsline{toc}{subsection}{Article VII. Démission d'un poste}
Si un membre démissionne d’un poste, celui-ci se voit restituer la totalité ou une partie de ses points, si un préavis a été remis au CE-A et qu'une transition adéquate a été effectuée. Si le membre a effectué une partie de son mandat avant sa démission, les points seront distribués entre ce dernier et le nouveau détenteur du poste. La distribution des points entre ses deux membres sera déterminée par le CE-A.