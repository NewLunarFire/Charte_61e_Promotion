
\section*{Règlement 9 - Relatif aux commandites} 
\addcontentsline{toc}{section}{Règlement 8 - Relatif aux Points Génie}

\subsection*{Article I. Le but du règlement}
\addcontentsline{toc}{subsection}{Article I. Le but du règlement}
Ce règlement a pour but de clarifier l'utilité de trouver des commandites, ce à quoi cet argent va servir, les modalités de remboursement et les récompenses qui s'y rattachent. 

\subsection*{Article II. Les commandites}
\addcontentsline{toc}{subsection}{Article II. Les commandites}
Les commandites sont de l’argent recueilli par les membres de la promotion qui servira à financer les activités de la promotion finissante. Les commanditaires ont la possibilité de donner un montant proportionnel à la grandeur de la publicité accordée dans l’album, le média électronique et tout autre article, selon un barème déterminé par la présidence. 

\subsection*{Article III. L'utilisation des fonds}
\addcontentsline{toc}{subsection}{Article III. L'utilisation des fonds}
L'argent amassé servira à financer les activités de la promotion finissante, l’album, ainsi que le média électronique. L’argent amassé en commandites devra être versé dans le compte courant de la promotion. La balance de la commandite non réclamée par le membre sera versée dans le compte commun de la promotion.

\subsection*{Article IV. Les récompenses}
\addcontentsline{toc}{subsection}{Article IV. Les récompenses}
Une récompense sera consentie et versée dans le compte personnel du membre pour le montant amassé en commandites, de la façon suivante :\\

Un montant pouvant aller jusqu’à 20\% du montant total de la commandite pourra être versé en commission dans le compte de l’avoir personnel du membre en question. Si le membre désire se prévaloir du montant partiel ou total de sa commission, il devra en informer la vice-présidence qui verra à inscrire cette somme à son compte personnel. Voir le règlement 6 pour plus de détails concernant l’avoir personnel des membres. \\

Dans le cas de commandites en biens, articles, rabais ou toute autre valeur qui n’est pas directement en argent, une récompense pourra être versée au membre si la commandite en question permet à la promotion de réaliser un profit et/ou une économie. Cette récompense sera déterminée par la vice-présidence et pourra aller jusqu’à 20\% de la valeur marchande réelle et courante du bien en question. 

