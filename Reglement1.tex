\section*{Règlement 1 - Relatif à la promotion} 
\addcontentsline{toc}{section}{Règlement 1 - Relatif à la promotion}

\subsection*{Article I. La dénomination sociale (nom)}
\addcontentsline{toc}{subsection}{Article I. La dénomination sociale (nom)}
\textbf{« La 61$^e$ promotion de génie de l’Université de Sherbrooke »}\\

Dans les règlements qui suivent, les termes « organisme » ou « promotion » désignent \textbf{« La 61$^e$ promotion de génie de l’Université de Sherbrooke »}. Le terme « université » désigne l’« Université de Sherbrooke ». L’acronyme « CE » désigne le comité exécutif de la promotion. L’acronyme « CE-A » désigne le conseil exécutif actif et l’acronyme « CE-P » désigne le conseil exécutif passif. L’acronyme « AG » désigne l’assemblée générale de la promotion.

\subsubsection*{Section 1.01 : Définition de la 61$^e$ promotion de génie}
\addcontentsline{toc}{subsubsection}{Section 1.01 : Définition de la 61$^e$ promotion de génie}
La 61$^e$ promotion de génie est la 61$^e$ cohorte à faire son entrée à la Faculté de Génie de l’Université de Sherbrooke depuis sa création en 1954. 

\subsubsection*{Section 1.02 : Définition de la promotion finissante}
\addcontentsline{toc}{subsubsection}{Section 1.02 : Définition de la promotion finissante}
La 61$^e$ promotion sera considérée comme une promotion finissante à l’automne 2018, lors de la passation des pouvoirs par l’AGEG de la 60$^e$ promotion à la 61$^e$ promotion, jusqu’à la passation des pouvoirs à la 62$^e$ promotion, à l’automne 2019. 

\subsubsection*{Section 1.03 : Définition de programme actif}
\addcontentsline{toc}{subsubsection}{Section 1.03 : Définition de programme actif}
Tous les groupes d’un programme en session durant la session active sont considérés comme un programme actif. Tous les programmes qui ne sont pas en session sont des programmes passifs.

\subsubsection*{Section 1.04 : Définition de signataire de la promotion}
\addcontentsline{toc}{subsubsection}{Section 1.04 : Définition de Signataire}
Personne autorisée à signer des documents sous sa responsabilité, incluant les formulaires de dépôt de facture et de dépôt d'argent transmis à la coordonnatrice administrative/technicienne comptable de l'AGEG et les contrats.

\subsection*{Article II. Objets}
\addcontentsline{toc}{subsection}{Article II. Objets}
Promouvoir et défendre les intérêts intellectuels, culturels, académiques, sportifs, sociaux et matériels des membres de la promotion tel que décidé et voté par le CE.

\subsection*{Article III. Le logo}
\addcontentsline{toc}{subsection}{Article III. Le logo}
Le logo de \textbf{« La 61$^e$ promotion de génie de l’Université de Sherbrooke »}, ne doit pas être utilisé pour des fins d’articles promotionnels sans l’autorisation écrite du CE. Le logo actuel de la promotion est illustré à la figure 1.

\begin{figure}[H]
\begin{center}
\includegraphics[width=6cm]{LOGO61.png}
\caption{Logo de la 61$^e$ promotion de génie}
\end{center}
\end{figure}

\subsection*{Article IV. Association à l'AGEG}
\addcontentsline{toc}{subsection}{Article IV. Association à l'AGEG}
Toute décision prise par le CE ou par l’AG des membres est assujettie aux règlements en vigueur à l’AGEG.

\subsection*{Article V. Les membres}
\addcontentsline{toc}{subsection}{Article V. Les membres}

\subsubsection*{Section 5.01 : Les membres réguliers}
\addcontentsline{toc}{subsubsection}{Section 5.01 : Les membres réguliers}
Toute personne qui se juge admissible comme membre, mais ne respectant pas à la lettre les critères suivants peut en faire la demande par écrit auprès du CE actif.\\

Sont membres réguliers de la promotion, à moins d'une suspension, d'une expulsion ou d'une démission, toutes les personnes étudiantes :
\begin{enumerate}
\item ayant débuté leur programme de baccalauréat (d’un minimum de 120 crédits) à la Faculté de Génie de l'Université de Sherbrooke au plus tard à l’automne 2015
\item ayant réalisé plus de 60 crédits de baccalauréat avec la promotion 61
\item terminant leur baccalauréat avant le 1er septembre 2020
\end{enumerate}

\subsubsection*{Section 5.02 : Les membres non-réguliers (membre quittant vers la promotion suivante)}
\addcontentsline{toc}{subsubsection}{Section 5.02 : Les membres non-réguliers}
Sont membres non-réguliers de la promotion, à moins d'une suspension, d'une expulsion ou d'une démission, tous les personnes étudiantes ayant effectué moins de 60 crédits avec la promotion 61 qui respectent les critères d’une des deux catégories suivantes : 

\begin{enumerate}
\item ayant débuté leur programme de baccalauréat (d’un minimum de 120 crédits) à la Faculté de Génie de l'Université de Sherbrooke à l’automne 2015 avec la 61e promotion
\item étant inscrit en date du 1er novembre 2018 dans un programme de baccalauréat (d’un minimum de 120 crédits) à la Faculté de Génie de l’Université de Sherbrooke
\item terminant leur baccalauréat après le 1er septembre 2020
\end{enumerate}

\subsubsection*{Section 5.03 : Inscription au registre des membres}
\addcontentsline{toc}{subsubsection}{Section 5.03 : Inscription au registre des membres}

L’inscription des membres se fait en complétant le formulaire disponible auprès du CE. Au cours des sessions d’hiver 2018 et d’été 2018, un courriel sera envoyé via la liste promo-61-genie-request\\@listes.usherbrooke.ca, via le groupe Facebook de la promotion et via l’AGEGNews afin d’inviter les membres à s’inscrire. 


\subsubsection*{Section 5.04 : La suspension et l'expulsion}
\addcontentsline{toc}{subsubsection}{Section 5.04 : La suspension et l'expulsion}
L'AG pourra, en dernier recours, par simple résolution, suspendre pour la période qu'elle déterminera ou expulser définitivement tout membre dont la conduite ou les activités sont jugées nuisibles à la promotion. La décision de l'AG à cette fin sera finale et sans appel. Également, tout membre suspendu ou expulsé par l’AGEG sera immédiatement suspendu de la promotion. Dans le cas d’une expulsion par l’AGEG, il sera suspendu ne pourra plus occuper de certains postes au sein de la promotion (CE, comité 5@8 et bénévole), tout en conservant ses avantages acquis jusqu’à date au sein de la promotion (Points Génie). 

\subsubsection*{Section 5.05 : Le retrait}
\addcontentsline{toc}{subsubsection}{Section 5.05 : Le retrait}
Tout membre qui le désire peut se retirer de la promotion en envoyant un avis écrit à cet effet à la présidence du CE-A et au président de l'AGEG. 

\subsubsection*{Section 5.06 : Membre actif}
\addcontentsline{toc}{subsubsection}{Section 5.06 : Membre actif}
Un membre actif est une personne étudiante faisant partie de la 61$^e$ promotion, étant inscrit à temps partiel ou à temps plein à l’université et qui est en session d’étude sur le campus principal de l'Université de Sherbrooke (voir les alternances à l’annexe A).

\subsubsection*{Section 5.07 : Membre passif}
\addcontentsline{toc}{subsubsection}{Section 5.07 : Membre passif}
Un membre passif est un personne étudiante faisant partie de la 61$^e$ promotion, qui est en stage ou qui n’est ni une personne étudiante à temps plein ni une personne étudiante à temps partiel à l’université au campus principal de l'Université de Sherbrooke.